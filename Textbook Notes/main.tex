\documentclass[notitlepage]{article}
%\usepackage{lmodern}
%\usepackage[T1]{fontenc}
%\usepackage[spanish]{babel}
\usepackage[utf8]{inputenc}
\usepackage{amsmath}
\usepackage{amsfonts}
\usepackage{amssymb}
\usepackage{amsthm}
\usepackage{hyperref}
\author{David Cardozo}
\title{Textbook notes on Abstract Algebra}

\newtheorem{thm}{Theorem}
\newtheorem{lem}[thm]{Lemma}
\newtheorem{prop}{Proposition}[section]

\theoremstyle{definition}
\newtheorem{define}{Definition}[section]
\newtheorem{defn}{Definition}[section]
\newtheorem{conj}{Conjecture}[section]
\newtheorem{exm}{Example}[section]
\theoremstyle{remark}
\newtheorem*{rem}{Remark}
\newtheorem*{note}{Note}
\newtheorem{case}{Case}
\newtheorem{exc}{Exercise}
\newtheorem*{sol}{Solution}




%Customized Commands
\newcommand{\lrp}[1]{\left( #1 \right)}
\newcommand{\abs}[1]{\left| #1 \right|}
\newcommand{\set}[1]{\left\lbrace #1 \right\rbrace}
\newcommand{\RR}{\mathbb{R}}
\newcommand{\CC}{\mathbb{C}}
\newcommand{\QQ}{\mathbb{Q}}
\newcommand{\ZZ}{\mathbb{Z}}
\newcommand{\ZN}[1]{\frac{\mathbb{Z}}{#1 \mathbb{Z}}}
\newcommand{\PP}{\mathbb{P}}
\newcommand{\qt}[1]{\textrm{#1}}
\newcommand{\function}[3]{#1 : #2 \rightarrow #3}
\newcommand{\contained}{\subseteq}
\newcommand{\restric}[2]{ #1\restriction_{#2}}
\newcommand{\divs}{\mid}
\newcommand{\ndivs}{\nmid}
\newcommand{\normal}{\trianglelefteq}
\newcommand{\isomorphic}{\cong}
\newcommand{\inv}{^{-1}}
\newcommand{\subgroup}{\leq}
\newcommand{\centralizer}[2]{C_{#1}\lrp{#2}}
\newcommand{\normalizer}[2]{N_{#1}\lrp{#2}}
\newcommand{\acts}{\circlearrowright}
\newcommand{\powerset}[1]{\mathcal{P}\lrp{#1}}
\newcommand{\generatedcyclic}[1]{\left\langle #1 \right\rangle}
\newcommand{\order}[1]{\left| #1 \right|}

\begin{document}
\maketitle
The following are notes based on the book \textit{Abstract Algebra} by Dummit \& Foote.

\section{Basics}

We shall use the notation $ \function{f}{A}{B} $, and the value of $f$ at $a$ is denoted $ f(a) $, that is we shall apply our functions on the left). map is a synonymous of function. The set $A$ is called the domain of $f$ and $B$ the codomain of $f$. The notation $ a \rightarrowtail b$ if $f$ is understood indicates that $ f(a) = b$ 
The set 
\[  f(A) = \set{b \in B | b = f(a), \qt{ for some } a \in A} \]
is a subset of $ B $, called the \textbf{range } or \textbf{image} if f. Fir each subset $C$ of $B$ the set:
\[ f^{-1} (C) = \set{a \in A | f(a) \in C} \]
consisting of the elements of $A$ mapping into $C$ under $f$ the \textbf{preimage} or \textbf{inverse } image of C under f. For each $b \in B $, the preimage of $ \set{b} $ under $f$ is called the \textbf{fiber} of $f$ over $b$. The fibers of $f$ generally contain many elements since there may be many elements of $A$ mapping to the element $b$.

If $\function{f}{A}{B} $ and $\function{g}{B}{C} $, then the composite map $ \function{g \circ f}{A}{C} $ is defined by:
\[ \lrp{g \circ f}(a) = g(f(a))\]

Some important terminologies:
Let $\function{f}{A}{B} $:
\begin{itemize}
	\item f is \textbf{injective} if whenever $ a_1 \neq a_2 $, then $ f(a_1) \neq f(a_2) $
	\item $f$ is \textbf{surjective} if for all $b \in B$ there is some $a \in A$ such that $f(a) = b $
	\item $f$ is \textbf{bijective} or it is a bijection if it is both injective and surjective. 
	\item $f$ has a left inverse if there is a function $ \function{g}{B}{A} $ such that $ \function{ g \circ f}{A}{A} $ is the identity map.
	\item $f$ has a right inverse if there is function $ \function{h}{A}{B} $ such that $ \function{f \circ h}{B}{B} $ is the identity map on $B$. 
\end{itemize}
\begin{prop}
	Let $ \function{f}{A}{B} $.
	\begin{itemize}
		\item The map $f$ is injective if and only if $f$ has a left inverse.
		\item The map $f$ is surjective if and only if $f$ has a right inverse.
		\item The map $f$ is a bijection if and only if there exists $ \function{G}{B}{A} $ such that $f \circ g$ is the identity map on $B$ and $ g \circ f $ is the identity map on  $A$.
		\item If $A$ and $B$ are finite sets with the same number of elements. then $\function{f}{A}{B}$ is bijective if and only if $f$ is injective if and only if $f$ is surjective.
	\end{itemize}
\end{prop}

An important remark is that any function is surjective onto its range (by definition).

\begin{lem}
	The map $f$ is a bijection if and only if there exists $ \function{G}{B}{A} $ such that $f \circ g$ is the identity map on $B$ and $ g \circ f $ is the identity map on  $A$.
\end{lem}
\begin{proof}
	Suppose $f$ is a bijection, i.e, f is both surjective and injective. That is, since it is surjective, there exist $ \function{g}{B}{A} $ such that
	\[ f \circ g = 1_B \].
	Since it is injective, there exist a $ \function{g'}{B}{A} $ such that
	\[ g' \circ f = 1_A \]
	Now let us observe that $g = g'$. Take note that for any $b \in B$.
	\begin{align*}
		g(b) &= 1_A(g(b)) = (g' \circ f)(g(b)) \\
		&= ((g' \circ f)\circ g)(b) = (g'\circ(f \circ g))(b) \\
		&= (g'\circ 1_B)(b) \\
		&= g'(b)
	\end{align*}
\end{proof}
\begin{lem}
	If $A$ and $B$ are finite sets with the same number of elements. then $\function{f}{A}{B}$ is bijective if and only if $f$ is injective if and only if $f$ is surjective.
\end{lem}
\begin{proof}
	Suppose that $f$ is an injective function, then $f(A) = |A|$, this is known as the \textbf{cardinality of Image of Injection} and is proven using induction, therefore the subset $f(A)$ of $B$ has the same number of elements of $B$ and so $f(A) = B$, so $f$ is surjective, and this implies is a bijection.
\end{proof}

A \textbf{permutation} of a set $A$ is simply a bijection from $A$ to itself.
If $A \contained B$ and $ \function{f}{B}{C} $, we denote the \textbf{restriction} of $f$ to $A$ by $ \restric{f}{A} $
\section{Properties of the Integers}
\begin{itemize}
	\item \textbf{Well Ordering of $ \ZZ $} If $A$ is any nonempty subset of $ \ZZ^+ $, there is some element $ m \in A$ such that $ m \leq a$, for all $a \in A $.
	\item If $ a, b \in \ZZ $ with $ a \neq 0 $, we say a divides b if there is an element $ c \in \ZZ $ such that $ b = ac$. In this case we write	$ a \mid b$, otherwise we write $ x \nmid y $
	\item If $a,b \in \ZZ - \set{0} $, there is a unique positive integer $d$, called the \textbf{greatest common divisor } of $a$ and $b$, satisfying:
	\begin{itemize}
		\item $ d \divs a$, and $ d \divs b$, and
		\item If $ e \divs a$ and $ e \divs b$, then $ e \divs d$ 
	\end{itemize} 
	The notation for $d$ will be $ (a,b) $, if it happens that $ (a,b) = 1 $, we say that $a$ and $b$ are relatively prime.
	\item If $a,b \in \ZZ - \set{0} $, there is a unique positive integer $l$, called the \textbf{least common multiple} of $a$ and $b$ satisfying:
	\begin{itemize}
		\item $ a \mid l$ and $ b \mid l$, and
		\item if $ a \mid m $ and $b \mid m$, then $ l \divs m$ (so that $l$ is the least such multiple)
	\end{itemize}
	\item \textbf{The Division Algorithm}: if $a,b \in \ZZ $ and $ b \neq 0 $, then there exist unique $q,r \in \ZZ $ such that:
	\[ a = qb + r \quad \qt{and} \quad 0 \leq r < \abs{b}\]
	where $q$ is the quotient and $r$ is the remainder.
	\item \textbf{The Euclidean Algorithm} It produces the greatest common divisor of two integers.
	\item If $a,b \in \ZZ - \set{0} $, then there exist $x,y  \in \ZZ $ such that:
	\[ (a,b) = ax + by \]
	\item An element $p$ of $ \ZZ^+ $ is called a prime if $ p > 1$ and the only positive divisors of $p$ are $1$ and $p$.
	\item The \textbf{Fundamental Theorem of Arithmetic} If $n \in \ZZ$, $ n > 1 $, then $n$ can be factored uniquely into the product of primes.
	\item The Euler $\phi-\qt{function}$ is defined as: for $n \in \ZZ $ let $ \phi(n) $ be the number of positive integers $ a \leq n$ with $a$ relatively prime to $n$, i.e., $(a,n) =1$. For prime $p$, $ \phi(p) = p-1$, and more generally, for all $a \geq 1$ we have the formula:
	\[ \phi(p^a) = p^a - p^{a -1} = p^{a-1}(p-1)\]
	The function $ \phi $ is multiplicative in the sense that:
	\[ \phi(ab) = \phi(a)\phi(b) \quad \qt{ if } \quad (a,b) = 1\]	
\end{itemize} 
PUT LINE HERE!
\section{Exercises}
\textbf{3.} Prove that if $n$ is composite, then there are integers $a$ and $b$ such that $n$ divides $ab$ but $n$ does not divide either $a$ or $b$.
\textbf{Solution}
Since $n$ is composite, then $ n = ab$ with $a < n $ and $b < n$, in both cases, because $n$ cannot divide a positive number smaller than itself, so that $ n \divs n$ = $ n \divs ab$.

\textbf{6.} Prove the Well Ordering Property of $ \ZZ $ by induction and prove the minimal element is unique.
\begin{lem}
	Every nonempty subset $S \neq \varnothing \contained \ZZ^+ $ has a minimum. 
\end{lem}
\begin{proof}
	Let us define the set:
	\[ T = \set{n \in \ZZ^+ \cup \set{0} | n \leq s \qt{ for all } s \in S} \]
	Since $S \neq \varnothing $, we have that  $ T \neq \ZZ^+$, this is given by the fact that if $ s' \in S $, then $ s' + 1 \not\in T$.
	Observe that at most $ 0 \in T$ to be done!
\end{proof}

\textbf{7.} Prove that if $p$ is a prime, then $ \sqrt{p} $ is not a rational number
\begin{proof}
	Suppose $ \sqrt{p} $ is a rational number, in other words:
	\[ \sqrt{p} = \frac{a}{b} \quad \qt{with} \quad (a,b) = 1\]
	or equivalently:
	\[ b^2p = a^2 \]
	so we can see that $ p \divs a \cdot a $, and we can conclude that $ p \divs a$, i.e., $ a = kp$ for some integer $p$.
	returning to our previous expression, we have that:
	\[ b^2p = k^2p^2 \]
	so that:
	\[ b^2 = k^2p \]
	from which we conclude that $ p \divs b$, but this is a contradiction since $ (a,b) = 1$. Therefore, our assumption that $\sqrt{p} $ is a rational number must be wrong.
\end{proof}

\textbf{8.} Find a formula for the largest power of $p$ which divides $n!$
\url{https://www.proofwiki.org/wiki/Factorial_Divisible_by_Prime_Power}

\textbf{11.} To be asked also.

\section{Subgroups}

\begin{defn}
	Let $ G $ be a group. The subset $ H $ of $ G  $ is a \emph{subgroup} of $ G $ if $ H $ is nonempty and $ H $ is closed under products and inverses (i.e, $ x,y \in H \implies x\inv \in H $ and $ xy \in H $). If $ H $ is a subgroup of $ G $ we shall write $ H \leq G $
\end{defn}

\begin{prop}
	(\textbf{The Subgroup Criterion}) A subset $ H $ of a group $ G $ is a subgroup if and only if:
	\begin{itemize}
		\item $ H \neq \varnothing $, and
		\item for all $ x,y \in H \implies x \cdot y\inv \in H $
	\end{itemize}
	Furthermore, if $ H $ is finite, then it suffices to check that $ H $ is nonempty and closed under multiplication.
\end{prop}
\begin{proof}
	Suppose $ H $ is a subgroup of $ G $, then is certain that (1) and (2)  holds, since it contains the identity of $ G $ and the inverse of each of its elements and because $ H $ is closed under multiplication.
	It remains to show conversely that if $ H $ satisfies both (1)  and (2), then $ H \subgroup G $. Let $ x $ be any element in $ H $. Let $ y =x $ and apply property (2) to deduce that $ 1 = xx\inv \in H $ so $ H $ contains the identity of $ G $. Then, again by (2), since $ H $ contains $ 1 $ and $ x $, $ H $ contains the element $ 1x\inv  $, that is$ x\inv \in H $ and $ H  $ is closed under taking inverses. Finally, if $ x $ and $ y $ are any two elements of $ H $, then $ H $ contains $ x $ and $ y \inv $, so by (2), $ H $ also contains $ x(y\inv)\inv $ that is $ xy $ Hence $ H $ is also closed under multiplication, which proves $ H $ is a subgroup of $ G $.
\end{proof}

\subsection{Centralizers and Normalizers, Stabilizers and Kernels}
We introduce some important families of subgroups of an arbitrary group $ G $.

\begin{defn}
	Let $ A $ be any nonempty subset of $ G $. Define \[  C_G(A)  = \set{g \in G | gag\inv = a \textrm{ for all } a \in A}  \]This subset of $ G $ is called the \emph{centralizer} of $ A $ in $ G $. Since $ gag\inv = a $ if and only if $ ga = ag $, $ C_G(A) $ is the set of elements of $ G $ which commute with every element of $ A $
\end{defn}

\begin{prop}
	The centralizer $ C_G(A) $ is a subgroup.
\end{prop}
\begin{proof}
	First we show that is nonempty. Let us observe that $ 1 \in C_G(A) $ since $ 1a1\inv = a $, so that $ C_G(A) \neq \varnothing $. Now let $ x, y \in \centralizer{G}{A} $, so that $ xax\inv =a $ and $ yay\inv = a $, observe that since $ yay\inv = a $, multiplying wisely in both left and right we have $ y\inv a y = a $  so that $ y\inv \in \centralizer{G}{A} $ so now let us consider:
	\begin{align*}
	(xy)a(xy)\inv = (xy)a(y\inv x\inv) \\
	&= x(yay\inv)x\inv \\
	&= x a x\inv \\
	&= a
	\end{align*}
	so that $ \centralizer{G}{A} $ is closed under product and taking inverses so that $ \centralizer{G}{A} \subgroup G $. 
\end{proof}
In the special case that $ A = \set{a} $ we write $ \centralizer{G}{a} $, observe that $ a^n \in \centralizer{G}{A} $ for all $ n \in \ZZ $

\begin{defn}
	Define $ Z(G) = \set{g \in G| gx = xg \textrm{ for all } x \in G} $, the set of elements commuting with all the elements of $ G $. This subset of $ G $ is called the \textit{center} of $ G $.
\end{defn}
\begin{rem}
	The center of a group is a subgroup
\end{rem}
\begin{proof}
	The center of a group is an special case of the centralizer since $ Z(G) = \centralizer{G}{G} $
\end{proof}
\begin{defn}
	Define $ gAg\inv = \set{gag\inv | a \in A} $. Define the \textbf{normalizer} of $ A $ in $ G $ to be the set $ \normalizer{G}{A} = \set{g \in G | gAg\inv = A} $.
\end{defn}
Let us remark that if $ g \in \centralizer{G}{A} $ then $ gag^-1 =a \in  A $ so that $ \centralizer{G}{A} \subgroup \normalizer{G}{A} $

\subsection{Stabilizers and Kernels of Group Actions}
We can indicate that the structure of $ G $ is reflected by the sets on which it acts, as follows: if $ G $ is a group acting on a set $ S $ and $ s $ is ome fixed element of $ S $, the \emph{stabilizer} of $ s $ in G is the set:
\[ G_s = \set{g \in G | g \cdot s = s} \]

\begin{exc}
	Let $ G $ be a group acting on a set $ A $ and fix some $ a \in A $. Show that the following sets are subgroups of $ G $:
	\begin{enumerate}
		\item the kernel of the action,
		\item $ \set{g \in G | ga = a} $ this subgroup is called the \textit{stabilizer} of $ a $ in $ G $.
	\end{enumerate}
\end{exc}
\begin{sol}
	\begin{itemize}
		\item So let $ G \acts A $, consider $ \ker(G \acts A) $, we want to see that it is in fact a subgroup. First observe that is nonempty since $ 1 \cdot a = a  $ for all $ a \in A $ so that $ 1 $ belongs to the kernel, now let $ x,y $ belong to the kernel, observe that since $ y  $ is in the kernel $ y \cdot a = a $ for all $ a \in A $. Now:
		\begin{align*}
		e \cdot s = s \\
		(g\inv \star g) s = s \\
		g\inv \cdot (g\cdot s) = s \\
		g\inv \cdot (s) = s
		\end{align*}
		and we observe that $ g\inv $ belongs to the kernel, so that the set is closed under inverses, for multiplication let us observe:
		\begin{align*}
		(x \star y) \cdot s = x \cdot ( y \cdot s) \\
		&= x \cdot (s) 
		&= s
		\end{align*}
		so that is closed under multiplication.
		\item As above, the same procedure holds but only for a member of $ s $ which does not changes the argument above.
	\end{itemize}
	
\end{sol}

Finally, we observe that the fact that centralizers, normalizers and kernels are subgroups is a special case of the facts that stabilizers and kernels of actions are subgroups. Let $ S = \mathcal{P}(G) $, the collection of all subsets of $ G $, and let $ G $ act on $ \powerset{G} $ by \emph{conjugation} $ g \in G $ and $ B \in \powerset{G}, B \subset G $:
\[ g: B \rightarrow gBg\inv \]
under this action, we see that the normalizer ($ \normalizer{G}{A} $) is precisely the stabilizer of $ A $ in $ G $, $ G_s = \normalizer{G}{A} $, where $ s = A \in \powerset{G} $, so that $ \normalizer{G}{A} $ is a subgroup of $ G $.
\begin{exc}
	Let $ G $ be any group and let $ A = G $. Show that the maps defined by $ g \cdot a = gag\inv $ for all $ a,g \in G $ satisfy the axioms of a (left) group action.
\end{exc}
\begin{sol}
	So let $ G \acts A $, and $ A = G $ via: $ g \cdot a = gag\inv $, let us observe that $ 1 \cdot a = 1a1\inv = a $ so that the first condition holds. Secondly, observe 
	\begin{align*}
	 x \cdot ( y\cdot a) = x \cdot (yay\inv)  \\
	 &= x(yay\inv)x\inv
	 &= (xy)a(xy)\inv
	\end{align*}
	so that the axioms for an actions are satisfied.
\end{sol}
Next let the group $ \normalizer{G}{A} $ act on the set $ S = A $ by conjugation, that is for all $ g \in \normalizer{G}{A} $ and $ a \in A $
\[ g: a \rightarrowtail gag\inv \]
Note that this does map $ A $ to $ A $ by the definition $ \normalizer{G}{A} $ and so gives an action on $ A $. The Kernel of this action is precisely $ \centralizer{G}{A} $ hence $ \centralizer{G}{A} \subgroup \normalizer{G}{A} $. Finally $ Z(G) $ is the kernel of $ G $ action on $ S = G $ by conjugation, so $ Z(G) \leq G $

\subsection{Cyclic Groups and Cyclic Subgroups}

\begin{defn}
	A group $ H $ is \emph{cyclic} if $ H $ can be generated by a single element, i.e., there is some element $ x \in H $ such that $ H = \set{x^n | n \in \ZZ} $ (where as usual the operation is multiplication).
\end{defn}
In additive notation $ H $ is cyclic if $ H = \set{nx | n \in Z} $. In both cases will write $ H = \generatedcyclic{x} $, we observe that $ H = \generatedcyclic{x} = \generatedcyclic{x\inv} $ so that it may have more than one generator. \textbf{by the law of exponents cyclic groups are abelian}.

\begin{prop}
	If $ H = \generatedcyclic{x} $, then $ \abs{H} = \abs{x} $(where if one side of this equality is infinite, so is the other). More specifically:
	\begin{itemize}
		\item if $ \order{H} = n < \infty $, then $ x^n = 1$, and $ 1,x,x^2, \ldots,x^{n-1} $ are all the distinct elements of $ H $, and:
		\item if $ \order{H} = \infty $, then $ x^n \neq 1 $ for all $ n \neq 0 $ and $ x^{a} \neq x^b $ for all $ a \neq b $ in $ \ZZ $.
	\end{itemize}
\end{prop}
\begin{proof}
	Let $ \order{x} = n$ and consider the finite case. The elements $ 1,x,x^2, \ldots, x^{n-1} $ are distinct because if $ x^a = x^b  $, with say $ 0 \leq a \leq b < n $, then $ x^{b-a} = x^0 =1 $. Contrary to the hypothesis that $ n $ was the smallest positive power of x that equals 1. This $ H $ has at least $ n $ elements and it remains to show that these are all of them. Let $ x^t $ is any power of $ x $, we use the Division Algorithm to write $ t = nq + k  $ where $  0 \leq k < n $, so:
	\[ x^t = x^{nq + k}  = x^{nq}x^k = 1x^k = x^k \in \set{1,x,\ldots,x^{n-1}}\]
	
	For the infinite case, observe then that no positive power of $ x $ is the identity. If $ x^a = x^b $ for some $ a $ and $ b $ then $ x^{a-b} = 1 $, which contradicts our hypothesis. So we conclude that distinct power of $ x $ are distinct elements of $ H $ so $ \order{H} = \infty $
\end{proof}
Observe that the calculations of distinct powers of a generator of a cyclic group of order n are carried out via arithmetic $ \ZN{n} $, the following reasoning proves that the groups are isomorphic.

\begin{prop}
	Let $ G $ be an arbitrary group, $ x \in G $ and let $ m,n \in \ZZ $. If $ x^n = 1 $ and $ x^m = 1 $, then $ x^d = 1 $, where $ d = (m,n) $. In particular, if $ x^m = 1 $ for some $ m \in \ZZ $, then $ \order{x} $ divides $ m $
\end{prop}
\begin{proof}
	Consider $ d = (m,n) $ by the Euclidean Algorithm there exists integers $ r,s $ for which $ d = rm + sn $, and $ d $ is the greates common divisor of $ m $ and $ n $. Thus:
	\[ x^d = x^{mr + ns} = x^{mr}x^{ns} = 1 \]
	This proves the first assertion.
	For the second assertion if $ x^m = 1 $, and let $ n = \order{x} $. If $ m = 0 $, certainly $ n \divs m $, so assume $ m \neq 0 $. Since some nonzero power of $ x $ is the identity, $ n \leq \infty $. Let $ d = (m,n) $ so by the same observation above:
	\[ x^d = 1 \]
	Since $ 0 < d \leq n $ and $ n $ is the smallest positive positive power of $ x $ which gives the identity, we must have $ d = n $, that is, $ n \divs m $ as asserted.
\end{proof} 

\section{Direct and Semidirect Products and Abelian Groups}

\subsection{Direct Products}
\begin{defn}
The \textbf{direct product} $ G_1 \times G_2 \times \ldots \times G_n $ of the groups $ G_1, G_2, \ldots, G_n $ with operations $ \star_1, \star_2, \ldots, \star_n $, respectively, is the set of n-tuples $ (g_1, \ldots, g_n) $ where $ g_i \in G_i $ with operation defined :
\[ (g_1, \ldots, g_n) \star (h_1, h_2, \ldots, h_n) = (g_1 \star_1 h_1, \ldots, g_n \star_n h_n). \]
\end{defn}
Similarly:

\begin{defn}
	The \textbf{direct product} $ G_1 \times \ldots $ of the groups $ G_1,G_2, \ldots $ with operations $ \star_1, \ldots $ respectively, is the set of sequences $ (g_1, g_2, \ldots) $ where $ g_i \in G_i $ with operation defined componentwise:
	\[ (g_1, g_2, \ldots) \star (h_1,h_2, \ldots) = (g_1 \star_1 h_1, \ldots). \]
\end{defn}
\begin{prop}
	If $ G_1, \ldots, G_n $ are groups, their direct product is a group of order $ \abs{G_1} \cdots \abs{G_n} $ (if any $ G_i $ is infinite, so is the direct product).	
\end{prop}
\begin{proof}
	Prove that it is a group (each of the axiom of a group holds componentwise) and a counting argument should hold.
\end{proof}
\begin{prop}
	Let $ G_1, G_2, \ldots, G_n $ be groups and let $ G = G_1 \times \cdots \times G_n $ be their direct product.
	\begin{enumerate}
		\item For each fixed $ i $ the set of elements of $ G $ which have the identity of $ G_j $ in the $ j^{\textrm{th}} $ position for all $ j \neq i $ and arbitrary elements of $ G_i  $ in position $ i $ is a subgroup of $ G $ isomorphic to $ G_i $:
		\[ G_i \cong \set{(1,1, \ldots, 1, g_i,1, \ldots,1) | g_i \in G_i} \]
		If we identify $ G_i $ with this subgroup, then $ G_i \normal G $ and:
		\[ G/G_i \isomorphic G_1 \times \cdots G_{i-1} \times G_{i+1} \times \cdots \times G_n \]
		\item For each fixed $ i  $ define $ \pi_i: G \rightarrow G_i $ by:
		\[ \pi_i((g_1,\ldots,g_n)) = g_i \]
		Then $ \pi_i $ is a surjective homomorphism with:
		\[ \ker\pi_i = \set{(g_1, \ldots,g_{i-1},1,g_{i+1}, \ldots, g_n) | g_j \in G_j \textrm{ for all } j \neq i} \]
		\[ \isomorphic G_1 \times \cdots \times G_{i-1} \times G_{i+1} \times G_n \]
		\item Under the identifications in part $ (1) $, if $ x \in G_i $ and $ y \in G_j $ for some $ i \neq j $, then $ xy = yx $
	\end{enumerate}
	
\end{prop}

	
\end{document}