\section{Basics}

We shall use the notation $ \function{f}{A}{B} $, and the value of $f$ at $a$ is denoted $ f(a) $, that is we shall apply our functions on the left). map is a synonymous of function. The set $A$ is called the domain of $f$ and $B$ the codomain of $f$. The notation $ a \rightarrowtail b$ if $f$ is understood indicates that $ f(a) = b$ 
The set 
\[  f(A) = \set{b \in B | b = f(a), \qt{ for some } a \in A} \]
is a subset of $ B $, called the \textbf{range } or \textbf{image} if f. Fir each subset $C$ of $B$ the set:
\[ f^{-1} (C) = \set{a \in A | f(a) \in C} \]
consisting of the elements of $A$ mapping into $C$ under $f$ the \textbf{preimage} or \textbf{inverse } image of C under f. For each $b \in B $, the preimage of $ \set{b} $ under $f$ is called the \textbf{fiber} of $f$ over $b$. The fibers of $f$ generally contain many elements since there may be many elements of $A$ mapping to the element $b$.

If $\function{f}{A}{B} $ and $\function{g}{B}{C} $, then the composite map $ \function{g \circ f}{A}{C} $ is defined by:
\[ \lrp{g \circ f}(a) = g(f(a))\]

Some important terminologies:
Let $\function{f}{A}{B} $:
\begin{itemize}
	\item f is \textbf{injective} if whenever $ a_1 \neq a_2 $, then $ f(a_1) \neq f(a_2) $
	\item $f$ is \textbf{surjective} if for all $b \in B$ there is some $a \in A$ such that $f(a) = b $
	\item $f$ is \textbf{bijective} or it is a bijection if it is both injective and surjective. 
	\item $f$ has a left inverse if there is a function $ \function{g}{B}{A} $ such that $ \function{ g \circ f}{A}{A} $ is the identity map.
	\item $f$ has a right inverse if there is function $ \function{h}{A}{B} $ such that $ \function{f \circ h}{B}{B} $ is the identity map on $B$. 
\end{itemize}
\begin{prop}
	Let $ \function{f}{A}{B} $.
	\begin{itemize}
		\item The map $f$ is injective if and only if $f$ has a left inverse.
		\item The map $f$ is surjective if and only if $f$ has a right inverse.
		\item The map $f$ is a bijection if and only if there exists $ \function{G}{B}{A} $ such that $f \circ g$ is the identity map on $B$ and $ g \circ f $ is the identity map on  $A$.
		\item If $A$ and $B$ are finite sets with the same number of elements. then $\function{f}{A}{B}$ is bijective if and only if $f$ is injective if and only if $f$ is surjective.
	\end{itemize}
\end{prop}

An important remark is that any function is surjective onto its range (by definition).

\begin{lem}
	The map $f$ is a bijection if and only if there exists $ \function{G}{B}{A} $ such that $f \circ g$ is the identity map on $B$ and $ g \circ f $ is the identity map on  $A$.
\end{lem}
\begin{proof}
	Suppose $f$ is a bijection, i.e, f is both surjective and injective. That is, since it is surjective, there exist $ \function{g}{B}{A} $ such that
	\[ f \circ g = 1_B \].
	Since it is injective, there exist a $ \function{g'}{B}{A} $ such that
	\[ g' \circ f = 1_A \]
	Now let us observe that $g = g'$. Take note that for any $b \in B$.
	\begin{align*}
		g(b) &= 1_A(g(b)) = (g' \circ f)(g(b)) \\
		&= ((g' \circ f)\circ g)(b) = (g'\circ(f \circ g))(b) \\
		&= (g'\circ 1_B)(b) \\
		&= g'(b)
	\end{align*}
\end{proof}
\begin{lem}
	If $A$ and $B$ are finite sets with the same number of elements. then $\function{f}{A}{B}$ is bijective if and only if $f$ is injective if and only if $f$ is surjective.
\end{lem}
\begin{proof}
	Suppose that $f$ is an injective function, then $f(A) = |A|$, this is known as the \textbf{cardinality of Image of Injection} and is proven using induction, therefore the subset $f(A)$ of $B$ has the same number of elements of $B$ and so $f(A) = B$, so $f$ is surjective, and this implies is a bijection.
\end{proof}

A \textbf{permutation} of a set $A$ is simply a bijection from $A$ to itself.
If $A \contained B$ and $ \function{f}{B}{C} $, we denote the \textbf{restriction} of $f$ to $A$ by $ \restric{f}{A} $
\section{Properties of the Integers}
\begin{itemize}
	\item \textbf{Well Ordering of $ \ZZ $} If $A$ is any nonempty subset of $ \ZZ^+ $, there is some element $ m \in A$ such that $ m \leq a$, for all $a \in A $.
	\item If $ a, b \in \ZZ $ with $ a \neq 0 $, we say a divides b if there is an element $ c \in \ZZ $ such that $ b = ac$. In this case we write	$ a \mid b$, otherwise we write $ x \nmid y $
	\item If $a,b \in \ZZ - \set{0} $, there is a unique positive integer $d$, called the \textbf{greatest common divisor } of $a$ and $b$, satisfying:
	\begin{itemize}
		\item $ d \divs a$, and $ d \divs b$, and
		\item If $ e \divs a$ and $ e \divs b$, then $ e \divs d$ 
	\end{itemize} 
	The notation for $d$ will be $ (a,b) $, if it happens that $ (a,b) = 1 $, we say that $a$ and $b$ are relatively prime.
	\item If $a,b \in \ZZ - \set{0} $, there is a unique positive integer $l$, called the \textbf{least common multiple} of $a$ and $b$ satisfying:
	\begin{itemize}
		\item $ a \mid l$ and $ b \mid l$, and
		\item if $ a \mid m $ and $b \mid m$, then $ l \divs m$ (so that $l$ is the least such multiple)
	\end{itemize}
	\item \textbf{The Division Algorithm}: if $a,b \in \ZZ $ and $ b \neq 0 $, then there exist unique $q,r \in \ZZ $ such that:
	\[ a = qb + r \quad \qt{and} \quad 0 \leq r < \abs{b}\]
	where $q$ is the quotient and $r$ is the remainder.
	\item \textbf{The Euclidean Algorithm} It produces the greatest common divisor of two integers.
	\item If $a,b \in \ZZ - \set{0} $, then there exist $x,y  \in \ZZ $ such that:
	\[ (a,b) = ax + by \]
	\item An element $p$ of $ \ZZ^+ $ is called a prime if $ p > 1$ and the only positive divisors of $p$ are $1$ and $p$.
	\item The \textbf{Fundamental Theorem of Arithmetic} If $n \in \ZZ$, $ n > 1 $, then $n$ can be factored uniquely into the product of primes.
	\item The Euler $\phi-\qt{function}$ is defined as: for $n \in \ZZ $ let $ \phi(n) $ be the number of positive integers $ a \leq n$ with $a$ relatively prime to $n$, i.e., $(a,n) =1$. For prime $p$, $ \phi(p) = p-1$, and more generally, for all $a \geq 1$ we have the formula:
	\[ \phi(p^a) = p^a - p^{a -1} = p^{a-1}(p-1)\]
	The function $ \phi $ is multiplicative in the sense that:
	\[ \phi(ab) = \phi(a)\phi(b) \quad \qt{ if } \quad (a,b) = 1\]	
\end{itemize} 
PUT LINE HERE!
\section{Exercises}
\textbf{3.} Prove that if $n$ is composite, then there are integers $a$ and $b$ such that $n$ divides $ab$ but $n$ does not divide either $a$ or $b$.
\textbf{Solution}
Since $n$ is composite, then $ n = ab$ with $a < n $ and $b < n$, in both cases, because $n$ cannot divide a positive number smaller than itself, so that $ n \divs n$ = $ n \divs ab$.

\textbf{6.} Prove the Well Ordering Property of $ \ZZ $ by induction and prove the minimal element is unique.
\begin{lem}
	Every nonempty subset $S \neq \varnothing \contained \ZZ^+ $ has a minimum. 
\end{lem}
\begin{proof}
	Let us define the set:
	\[ T = \set{n \in \ZZ^+ \cup \set{0} | n \leq s \qt{ for all } s \in S} \]
	Since $S \neq \varnothing $, we have that  $ T \neq \ZZ^+$, this is given by the fact that if $ s' \in S $, then $ s' + 1 \not\in T$.
	Observe that at most $ 0 \in T$ to be done!
\end{proof}

\textbf{7.} Prove that if $p$ is a prime, then $ \sqrt{p} $ is not a rational number
\begin{proof}
	Suppose $ \sqrt{p} $ is a rational number, in other words:
	\[ \sqrt{p} = \frac{a}{b} \quad \qt{with} \quad (a,b) = 1\]
	or equivalently:
	\[ b^2p = a^2 \]
	so we can see that $ p \divs a \cdot a $, and we can conclude that $ p \divs a$, i.e., $ a = kp$ for some integer $p$.
	returning to our previous expression, we have that:
	\[ b^2p = k^2p^2 \]
	so that:
	\[ b^2 = k^2p \]
	from which we conclude that $ p \divs b$, but this is a contradiction since $ (a,b) = 1$. Therefore, our assumption that $\sqrt{p} $ is a rational number must be wrong.
\end{proof}

\textbf{8.} Find a formula for the largest power of $p$ which divides $n!$
\url{https://www.proofwiki.org/wiki/Factorial_Divisible_by_Prime_Power}

\textbf{11.} To be asked also.
