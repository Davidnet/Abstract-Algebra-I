\documentclass[notitlepage]{article}
%\usepackage{lmodern}
%\usepackage[T1]{fontenc}
%\usepackage[spanish]{babel}
\usepackage[utf8]{inputenc}
\usepackage{amsmath}
\usepackage{amsfonts}
\usepackage{amssymb}
\usepackage{amsthm}
\usepackage{hyperref}
\author{David Cardozo}
\title{ Abstract Algebra}

\newtheorem{thm}{Theorem}
\newtheorem{lem}[thm]{Lemma}
\newtheorem{prop}{Proposition}[section]

\theoremstyle{definition}
\newtheorem{define}{Definition}[section]
\newtheorem{defn}{Definition}[section]
\newtheorem{conj}{Conjecture}[section]
\newtheorem{exm}{Example}[section]
\theoremstyle{remark}
\newtheorem*{rem}{Remark}
\newtheorem*{note}{Note}
\newtheorem{case}{Case}
\newtheorem{exc}{Exercise}
\newtheorem*{sol}{Solution}




%Customized Commands
\newcommand{\lrp}[1]{\left( #1 \right)}
\newcommand{\abs}[1]{\left| #1 \right|}
\newcommand{\set}[1]{\left\lbrace #1 \right\rbrace}
\newcommand{\RR}{\mathbb{R}}
\newcommand{\CC}{\mathbb{C}}
\newcommand{\QQ}{\mathbb{Q}}
\newcommand{\ZZ}{\mathbb{Z}}
\newcommand{\ZN}[1]{\frac{\mathbb{Z}}{#1 \mathbb{Z}}}
\newcommand{\PP}{\mathbb{P}}
\newcommand{\qt}[1]{\textrm{#1}}
\newcommand{\function}[3]{#1 : #2 \rightarrow #3}
\newcommand{\contained}{\subseteq}
\newcommand{\restric}[2]{ #1\restriction_{#2}}
\newcommand{\divs}{\mid}
\newcommand{\ndivs}{\nmid}
\newcommand{\normal}{\trianglelefteq}
\newcommand{\isomorphic}{\cong}
\newcommand{\inv}{^{-1}}
\newcommand{\subgroup}{\leq}
\newcommand{\centralizer}[2]{C_{#1}\lrp{#2}}
\newcommand{\normalizer}[2]{N_{#1}\lrp{#2}}
\newcommand{\acts}{\circlearrowright}
\newcommand{\powerset}[1]{\mathcal{P}\lrp{#1}}
\newcommand{\generatedcyclic}[1]{\left\langle #1 \right\rangle}
\newcommand{\order}[1]{\left| #1 \right|}

\begin{document}
\maketitle

\begin{exc}
	Let $ H(F) $ be the Heisenberg group over the field $ F $. Find and explicit formula for the commutator $ [X,Y] $, where $ X,Y \in H(F) $.
\end{exc}

\begin{sol}
	Let us remember that in the first homework we found that the inverse for a member of the Heisenberg group over $ F $ was:
	Observe that since the determinant of any element in $H$ is one, then there exist a inverse matrix, \emph{i.e} it has a matrix inverse, more explicitly:
	\[ 
	\begin{pmatrix}
	1&a&b  \\ 
	0&1&c \\    
	0&0&1
	\end{pmatrix}^{-1} =
	\begin{pmatrix}
	1 & -a & ac-b \\
	0 & 1 & -c \\
	0 & 0 & 1
	\end{pmatrix} \]
	so that let $ x,y \in H(F) $ so that:
	\[ X = \begin{pmatrix}
	1&a&b  \\ 
	0&1&c \\    
	0&0&1
	\end{pmatrix}  \]
	and 
	\[ Y = \begin{pmatrix}
	1&d&e  \\ 
	0&1&f \\    
	0&0&1
	\end{pmatrix}  \]
	by definition of the commutator $ [X,Y] = X\inv Y \inv XY $
	so that:
	\[ [X,Y]  = \begin{pmatrix}
	1&-a&ac - b  \\ 
	0&1&-c \\    
	0&0&1
	\end{pmatrix}
	 \begin{pmatrix}
		1&-d&df - e  \\ 
		0&1&-f \\    
		0&0&1
	\end{pmatrix} XY \]
	and putting this matrix into Mathematica (although I did remeber how to compute matrices but the laziness work), we found that:
	\[ [X,Y] = \begin{pmatrix}
	1&0&af-cd \\
	0&1&0 \\
	0&0&1
	\end{pmatrix} \]
	which is a closed formula for the commutator 
\end{sol}

\begin{exc}
	Show that this operation on $  \check{G} $ makes this set an abelian group. [Hint: show that the trivial homomorphism is the identity and that $ \chi^{-1}(g) = \chi(g)^{-1}. $]
\end{exc}

\begin{sol}
	Consider $ A $ as an abelian group, and $ G $ a group, now consider the set of all homomorphism $ G \rightarrow A $ , so that we want to see and define a product on this set. We will do ut via pointwise multiplication, for identity observe that the trivial homomorphism is on the set described above, so that $ (1 \bar{A})(x) = (\bar{A} 1 )(x) = \bar{a}(x) $ and we have our identity, Now consider an arbitrary homomorphism, say $ \phi $ and let $ \theta $ be the inverse of $ \phi $ (observe that is a function of G evaluated at some argument ). so that $ \theta(xy) = \phi(x)\inv \phi(y)\inv = \theta(y) \theta(x)$, and by abelian property we have: $ \theta(x) \theta(y) $, which means that is an homomorphism, and by natural function properties we find that $ (\theta \phi) (x) = \phi(x)\inv \phi(x) = 1(x) $ the identity function for all $ x $ which in other words:  $ \theta = \phi\inv  $ so that we have identity and inverse, therefore a group. We want to see that is abelian so that consider:
	\[ (\phi \gamma )(x) = \phi(x)\gamma(x) = \gamma(x) \phi (x) \] as above, so that we get $  (\gamma \phi)(x) 
	 $ and it is abelian.
	 \end{sol}

\begin{exc}
	Write definitions of the graph homomorphism and isomorphism $ \gamma_1 \rightarrow \gamma_2 $, and automorphism of a graph.
\end{exc}
\begin{sol}
	\begin{defn}
		A \textit{graph } consists of a set of \textit{vertices } $ V(G) $ and a set of \textit{edges} $ E(G) $ represented by unordered pair of vertices. We define vertices $ x $ and $ y $ to be a adjacent of $ (x,y) \in E(G) $
	\end{defn}
	\begin{define}
		We define that a graph is \textit{complete} if every vertex is connected to every other vertex 
	\end{define}
	We define the notion of isomorphism as:
	\begin{defn}
		An \textit{isomorphism} between two graphs $ G $ and $ H $ as a bijective map $ f: G \rightarrow H $ with the property:
		\[ (x,y) \in E(G) \iff (f(x),f(y)) \in E(H) \]
	\end{defn}
	we weakening the above definition for make sense of a graph homomorphism 
	\begin{defn}
		A \textit{homomorphism} from a graph $ G $ to a graph $ H $ is defined as a mapping $ h: G \rightarrow H $ such that
		\[  (x,y) \in E(G) \implies (f(x),f(y)) \in E(H) \]
	\end{defn}
	In the end we want that preserves edges. Which leads to:
	\begin{defn}
		A graph G with 2 independent vertex set is a bipartite graph. 
	\end{defn}
	Finally we want to discuss the transformations of a graph onto itself which leads to the concept of:
	\begin{defn}
		An isomorphism from a graph $ G $ to itself is called an \textbf{automorphism}
	\end{defn}
\end{sol}

\end{document}
