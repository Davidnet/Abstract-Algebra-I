\section{Dihedral Group}
Geometric Group.
\[ D_{2n} = \set{\textrm{the group of symmetries of the ngon}} \]
\[ \abs{D_{2n} = 2n} \]
elements: n rotations through $ \theta = 0, \frac{2 \pi}{n}, 2 \frac{2 \pi}{n}, ..., (n-1) \frac{2 \pi }{n}$
 and n more which are reflections thorough vertices. and $n$ reflections thoriugh edges.
 Rotations through  $ \frac{2\pi}{n} = r $, there are $n$, and let $s$, ..  $ \abs{s} = n $. and $ s \neq r^i $ for any $i$.
 \begin{itemize}
 	\item $ s \neq r^j $ for any j.
 	\item $ s r^{i} \neq s^j $ for all $ 0 \leq i \neq j \leq n-1$
 	\item  $ rs = s r^{-1} $, more generally
 	\item $ r^i s = s r^{-i} $ for $ 0 \leq i \leq n$	
 \end{itemize}
 
 
 \begin{define}
 	$S \subset $ G, the subgroup generated by $S$, denoted $ < S > = $ the smallest subgroup containing S.And formall $ \bigcap_{S \subset H \textrm{ subgroup } } H $ which is the collection all finite products and inverse of elements of $S$
 \end{define}
 
 \begin{example}
 $ <r> $ in $D_{2n} $ is $ \set{r^i: i} $ which is exactly $ \frac{Z}{2 Z} $, meanwhile $ <s> = \frac{Z}{2Z} $ which $ <r,s> = D_{2n} $ 
 \end{example}
 
 Any equation that the generators satisfy is called a \textbf{relation}
 
 \textbf{Notation} Presentation with generators and relations.
 
 \[  G = <S | R_1, \ldots R_m \]
 
 \begin{example}
 	\[ D_{2n} = < r,s | r^n = s^2 =1, rs = sr^{-1} > \]
 \end{example}
 
 \begin{example}
Symmetries of a regular tetrahedron $ = 12 $ 
 \end{example}
 
\section{Symmetric Group}
Let $ \Omega $ be a set then $S_{\Omega} $ be the set of bijection from $\Omega \rightarrow \Omega$:
\[ S_{\Omega} = \set{\sigma: \sigma: \Omega \rightarrow \Omega } \]
\begin{align*}
\Omega = [n] = \set{1,2,....n} \\
S_n : = S_{[n]}
\textrm{cycle} \quad (a_1 \rightarrow a_2 \ldots ... a_m ) \in S_n \\
\textrm{otherwise} 
\end{align*}
We define elements $(i j)^{-1} = (ij) $ $(ijk)=(jki) $.
we observe  $ \abs{S_n} = n! $ 
\begin{example}
	$ \abs{S_3} = 6 $ and $S_3$ is not abelian. $S_n$ $ n \geq 3$ is nonabelian.
\end{example}
\textbf{Disjoint cycles commute}, \textbf{rearranging the elements inside a cycle doesnt change it}

\section*{Matrix Group}

\begin{define}
	A \textbf{field} is the smallest math structure in which we can perform addition, and multiplication and division by nonzero element. To be more precise, a field $F$ is a set with two operations $+$ and $ \times $, such that:
	\begin{itemize}
		\item $ a \cdot ( b + c ) = a \cdot b + a \cdot c$
		\item $F^{\times} = F - \set{0} $ all nonzero elements are invertible.  
	\end{itemize}
\end{define}
Given any field $F$, we can construct $\operatorname{GL_n}(F)$ this is the group of all the invertible matrices over $F$.
to do: How many elements do we have in $ \abs{GL_n}(F_p) $ for case 2 $(p^2 -1)(p^2 - p) $