\documentclass[11pt,letterpaper,twocolumn]{article}
\usepackage[utf8]{inputenc}
\usepackage{amsmath}
\usepackage{amsfonts}
\usepackage{amssymb}
\usepackage{graphicx}
\author{David Cardozo}
\date{March 2015}
\title{The Keeler's Theorem and Symmetric Group}
\begin{document}

\twocolumn[
\begin{@twocolumnfalse}
	\maketitle
	\begin{abstract}
		An episode of \textit{Futurama} features a two-body mind-switiching machine which will not work more than once on the same pair of bodies. The main mathematical question on the episode is:  ``\textit{Is it possible to get everyone back to normal using four or more bodies?}''. A writer for the episode Ken Keeler found an algorithm that can solve any mind-body permutation. We present in this report an overview of a presentation of Keeler's result and presenting a refinement that optimizes the above algorithm. 
	\end{abstract}
\end{@twocolumnfalse}
]

\section{Introduction}





\end{document}